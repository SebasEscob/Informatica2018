\documentclass[12pt,letterpaper,twocolumn]{article}

\usepackage{graphicx}
\topmargin=-1in
\evensidemargin=0in
\oddsidemargin=0in
\textwidth=6.5in
\textheight=9.0in
\headsep=0.25in

\begin{document}

\title{Hoja de trabajo No.3}
\author{Sébastien Escobar}
\maketitle

\subsection*{Informatica 1}
\renewcommand{\abstractname}{Ejercicio No.1 }
\begin{abstract}
Utilizando la siguiente definicion 
\[
        n\oplus m := \left\{
        \begin{array}{l l}
            m & \mbox{si } n=o \\
            n & \mbox{si } m=o \\
            s(i\oplus m) & \mbox{si } n=s(i) \\
        \end{array}
        \right.
    \]
Se toma como posibilidad la suma de los numeros naturales unarios siendo estos  tres [$s(s(s(0)))$] y cuatro [$s(s(s(s(0))))$.
\\
\\Tomamos la definicion anterior y sustituimos n por tres y m por cuatro
\\
\\Tomando en cuenta nuestro caso base donde n=0
\\
\\Resulta que  [$s(s(s(0)))$] + 0 =  [$s(s(s(0)))$]
\\Y que 0 +  [$s(s(s(s(0))))$] =  [$s(s(s(s(0))))$]
\\
\\ Ya que nuestra definicion implica que si n=s(i) donde s(i) es un numero natural unario.
\\
\\ Proseguimos a la sustitucion de nuestros numeros, aplicando la tercera regla
\\
\\Donde  [$s(s(s(0)))$] +  [$s(s(s(s(0))))$]
\\
\\Dando asi,
\\s(s(s(s(s(s(s(0)))))))

\end{abstract}

\renewcommand{\abstractname}{Ejercicio No.2}
\begin{abstract}
Definicion inductiva de la multiplicacion
A  ($\otimes$)  B = A, si B=1
\\
\\A  ($\otimes$)  B = B, si A=1


\end{abstract}

\renewcommand{\abstractname}{Ejercicio No.3}
\begin{abstract}

\end{abstract}

\renewcommand{\abstractname}{Ejercicio No.4}
\begin{abstract}

\end{abstract}

\end{document}