\documentclass[12pt,letterpaper,twocolumn]{article}

\usepackage{graphicx}
\topmargin=-1in
\evensidemargin=0in
\oddsidemargin=0in
\textwidth=6.5in
\textheight=9.0in
\headsep=0.25in

\begin{document}

\title{Hoja de trabajo No.3}
\author{Sébastien Escobar}
\maketitle

\subsection*{Informatica 1}
\renewcommand{\abstractname}{Ejercicio No.1 }
\begin{abstract}
Utilizando la siguiente definicion 
\[
        n\oplus m := \left\{
        \begin{array}{l l}
            m & \mbox{si } n=o \\
            n & \mbox{si } m=o \\
            s(i\oplus m) & \mbox{si } n=s(i) \\
        \end{array}
        \right.
    \]
Se toma como posibilidad la suma de los numeros naturales unarios siendo estos  tres [$s(s(s(0)))$] y cuatro [$s(s(s(s(0))))$.
\\
\\Tomamos la definicion anterior y sustituimos n por tres y m por cuatro
\\
\\Tomando en cuenta nuestro caso base donde n=0
\\
\\Resulta que  [$s(s(s(0)))$] + 0 =  [$s(s(s(0)))$]
\\Y que 0 +  [$s(s(s(s(0))))$] =  [$s(s(s(s(0))))$]
\\
\\ Ya que nuestra definicion implica que si n=s(i) donde s(i) es un numero natural unario.
\\
\\ Proseguimos a la sustitucion de nuestros numeros, aplicando la tercera regla
\\
\\Donde  [$s(s(s(0)))$] +  [$s(s(s(s(0))))$]
\\
\\Dando asi,
\\s(s(s(s(s(s(s(0)))))))

\end{abstract}

\renewcommand{\abstractname}{Ejercicio No.2}
\begin{abstract}
Definicion inductiva de la multiplicacion
\\
\\Apoyandonos en la definicion inductiva de la suma, podemos aplicar el siguiente modelo base para nuestra demostracion.
\begin{center}
A  ($\otimes$)  B = AB
\end{center}
Siendo A un numero natural unario
Que puede expresarse como el siguiente
\begin{center}
S(0), expresando 1
\end{center}
Por otro lado un segundo numero natural unario B, expresado como S(S(0)) = 2, para este caso particular
\\
\\Antes de seguir, definimos la multiplicacion, como la suma de A, B veces.
\\
\\Lo cual con nuestra demostracion quedaria como:
\begin{center}
Siendo A = S(0)
\end{center}
El cual no puede expresarse como la suma de sus componentes ya que el se conforma a si mismo
\begin{center}
Y, B = S(S(0))
\end{center}
Logrando expresarse como la suma de sus componentes como:
\begin{center}
B = S(S(0)) = S(0) + S(0) 
\end{center}
\begin{center}

\end{center}
\end{abstract}
\renewcommand{\abstractname}{|}
\begin{abstract}
Ahora nuestra expresion se veria de la siguiente manera
\begin{center}
S(0) ($\otimes$) S(0) + S(0)
\end{center}
Aplicando nuestra definicion intuitiva de la multiplicacion, podemos decir que A se multiplica por la composicion de B, que en este caso es S(0) + S(0), dando como resultado:
\begin{center}
S(0) $\otimes$ S(0) + S(0) $\otimes$ S(0)
\end{center}
Si razonamos efectivamente podemos pensar que la expresion S(0) $\otimes$ S(0). (Para efectos practicos lo haremos con manzanas)
\\
\\Segun nuestra definicion, se escucharia como una vez una manza. Lo que seria una manza. Expresado en nuestro caso como S(0)
\\
\\Dejando asi nuestra expresion como: Una vez una manzana, mas, una vez una manzana
\\
\\Si aplicamos nuestra definicion de suma, quedaria como S(0) + S(0) = S(S(0))
Resultando asi 2.
\end{abstract}

\renewcommand{\abstractname}{Ejercicio No.3}
\begin{abstract}
Ya que tenemos nuestra definicion de multiplicacion es momento de ver si es correcta:
\\
\\Primer caso, S(S(S(0))) $\otimes$ 0
\\Tomando nuestra definicion se escucharia como, 0 veces S(S(S(0)))
\\
\\Ya que 0 no puede descomponerse se queda como si mismo, segun nuestro axioma
\\
\\Para S(S(S(0))), podemos descomponerlo en, S(0) + S(0) + S(0)
\begin{center}
S(0) $\otimes$ 0 + S(0) $\otimes$ 0 + S(0) $\otimes$ 0
\end{center}
Si definimos S(0) $\otimes$ 0, como 0 veces el sucesor de 0,
\begin{center}
Quedaria como 0 + 0 + 0 = 0
\end{center}
\end{abstract}

\renewcommand{\abstractname}{|}
\begin{abstract}
Para nuestro segundo caso, S(S(S(0))) $\otimes$ S(0)
\\Empezamos haciendo una distribucion de S(S(S(0))) en:
\begin{center}
(S(0) + S(0) + S(0)) $\otimes$ S(0)
\end{center}
Ahora multiplicamos cada uno de los terminos del lado derecho por el del lado izquierdo, lo que quedaria como
\begin{center}
    S(0) $\otimes$ S(0) + S(0) $\otimes$ S(0) + S(0) $\otimes$ S(0)
\end{center}
Segun nuestra definicion, lo anterior se podria escuchar como una vez uno. 
Lo que daria uno
\begin{center}
    Para lo cual: S(0) + S(0) + S(0) 
    Por definicion de la suma esto es igual a S(S(S(0)))
\end{center}
\end{abstract}

\renewcommand{\abstractname}{|}
\begin{abstract}
El tercer caso dispone S(S(S(0))) $\otimes$ S(S(0))
Empezamos por desgolsar cada uno de los terminos
\begin{center}
    S(S(S(0))) = S(0) + S(0) + S(0)
    \\S(S(0)) = S(0) + S(0) 
\\|
\\Lo que, quedaria como:
\\(S(0) + S(0) + S(0)) $\otimes$ (S(0) + S(0))
\end{center}
\\|
Por definicion de la multiplicacion, lo anterior se escucharia como 2 veces 3
\\
\\En lenguaje de numeros naturales unarios
\\Y aplicando nuestra def de multiplicacion esto se veria como
\begin{center}
    S(0) $\otimes$ S(0) + S(0) $\otimes$ S(0) + S(0) $\otimes$ S(0) + S(0) $\otimes$ S(0) + S(0) $\otimes$ S(0) + S(0) $\otimes$ S(0) 
\end{center}
\end{abstract}

\renewcommand{\abstractname}{
\\|
\\Ejercicio No.4}
\begin{abstract}
Demostraremos utilizando induccion lo siguiente
\\
\\Primer caso, a + S(S(0)) 
\\
\\Tomando la definicion conocida de la suma
\\Y aplicando su tercer concepto siendo este 
\begin{center}
s(i\oplus m) & \mbox{si } n=s(i) \\
\end{center}
Basandonos en esta definicion podemos representar nuestro problema como:
\begin{center}
    a + S(0) + S(0)
    \\S(a) + S(0)
    \\S(S(a))
\end{center}
|
\\
\\Segundo caso, a $\otimes$ b = b $\otimes$ a
\\
\\Demostramos la primera parte por nuestra definicion de multiplicaicon
\\Lo cual se escucharia como b veces a
\\
\\Debido a que lo anterior no se puede desglosar en sus componentes
\begin{center}
Lo anterior se ve como: a $\otimes$ b = ab
\end{center}
Si ahora aplicamos la segunda parte, se escucharia como a veces b
\\
\\Por ser una variable no se puede descomponer por lo cual se veria como
\begin{center}
b $\otimes$ a = ba
\\Al igualar los terminos de cada lado de nuestra exprecion queda como:
\\
\\ab = ba, 
\\A $\otimes$ B = B $\otimes$ A
\end{center}
\end{abstract}

\renewcommand{\abstractname}{|}
\begin{abstract}
Tercer caso
\\\item{$n \otimes (m \otimes l)=(n\otimes m)\otimes l$}
\\\item{{\bf Caso base: $l=o$}
\\\item{$n\otimes (m\otimes o)=n \otimes m = (n \otimes)\otimes o$}
\\
\\Seguimos con el \item{{\bf Caso inductivo: $l=s(i)$}
\\
\item{{\bf Hipotesis inductiva: }Asumimos que $m\otimes (n\otimes i)=(m\otimes n)\otimes i$}
            \item{Sabemos que $n\otimes(m\otimes s(i))=n\otimes s(m\otimes i)=s(n\otimes(m\otimes i))$}
            \item{Sabemos que $s(n\otimes(m\otimes i))=s((n\otimes m)\otimes i)$ por la hipotesis inductiva}
            \item{Concluimos que $s((n\otimes m)\otimes i)=(n\otimes m)\otimes s(i)$}
\end{abstract}
\end{document}
