\documentclass[12pt,letterpaper,twocolumn]{article}

\usepackage{graphicx} 
\usepackage{multirow}
\usepackage{hyperref}
\usepackage{amsmath}
\usepackage{amssymb}
\usepackage{graphicx}


\topmargin=-1in
\evensidemargin=0in
\oddsidemargin=0in
\textwidth=6.5in
\textheight=9.0in
\headsep=0.25in

\begin{document}

\title{Hoja de trabajo No.2}
\author{Sébastien Escobar}
\maketitle
\renewcommand{\abstractname}{Ejercicio No.1 }
\begin{abstract}
Demostración de: 
\[
        \forall\ n.\ n^3\geq n^2
\] donde $n\in\mathbb{N}$

\subsection*{Por metodo inductivo}
Se demuestra como primer punto que:
\\ (El número 1) siendo esté el primer natural cumple con la propiedad P1.
\\
\\Partiendo de la suposición:
\\ (Hipótesis de Inducción) donde un numero natural k cumple con la propiedad: P(k)
\\
\\Procedemos a demostrar que, el número k+1 cumple con la propiedad: P(k+1). 
\\Con lo cual validamos la implicación P(k)=P
\subsection*{Demostración}
Reescribamos \[
       \ n^3\geq n^2
\] Se puede tomar un caso base, 
\\donde todo n es = 0.
\\Para lo cual se demuestra que:
\[
        \ 0^3\geq 0^2
\] lo que es igual a
\[
        \ 0\geq 0
\]
Debido a que el caso base no nos lleva a una solución óptima, procedemos a analizar cada uno de los terminos
de la desigualdad y comenzamos a descomponerlo en sus factores.
\\
\\Utilizando la propiedad P(k) escribimos los terminos como:
\[(n+1)*(n+1)^2 = n^3\]
\[(n+1)^2 = n^2\]
\\Luego por la propiedad de división, pasamos a dividir el factor comun a ambos lados de la desigualdad.
\\Con lo que quedaria como:
\[[(n+1)*(n+1)^2 \geq (n+1)^2]/(n+1)^2\]
Aplicando la regla correctamente llegamos a que:
\[(n+1) \geq 1\]
Siendo esta una desigualdad podemos restar de ambos lados el 1
\[n+1-1 \geq 1-1\]
Llegando así por metodo inductivo a
\[n \geq 0\]
\end{abstract}


\renewcommand{\abstractname}{Ejercicio No.2}
\begin{abstract}
\subsection*{Demostrar utilizando inducción la desigualdad de Bernoulli:}
\[
        \forall\ n.\ (1+x)^n\geq nx
\]
\\donde $n\in \mathbb{N}$, $x\in \mathbb{Q}$ y $x\geq -1$

\subsection*{Por metodo inductivo}
Debido a que ya se demostro la propiedad P(k) ahora puede utilizarse, pero antes de ello se utiliza el caso base.
Donde n = 0.
\\
\[(1+x)^0 \geq (0)x\] Y esto es igual a: \[1 \geq 0\]
Lo anterior nos demuestra que el caso es cierto aunque no se halla demostrado, por lo que proseguimos a hacer la demostración.
\\
\\ Sea P(k) nuestra propiedad principal, tomando la desigualdad y aplicando la propiedad esta queda como:
\[(1+x)^{(n+1)} \geq ((n+1)*x)+1\]
\\
Por propiedades de exponente, descomponemos \[(1+x)^{(n+1)}\] en \[(1+x)(1+x)^n.\]
\\
Mientras que por el otro lado, simplificamos \[ ((n+1)*x)+1\] en \[(nx+x)+1.\]
\\
Ya simplificada la desigualdad, desarrollamos el polinomio quedando como:
\[ 1+x^n+x^{n+1}+x \geq nx+ x+1\]
\\
Eliminamos terminos semejantes, por la propiedad de resta:
\[x^{n+1}+x^n \geq nx \]
\\Sustituimos por el caso base n=0.
Donde 
\[x^{(0)+1}+x^{(0)} \geq (0)x \]
Al sustituir quedamos con..
\[x+1 \geq 0\]
Restamos el 1 de ambos lados de la desigualdad.
\\ Quedando así con:
\[ x \geq -1 \]


\end{abstract}

\end{document}