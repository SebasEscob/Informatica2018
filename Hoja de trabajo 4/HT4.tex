\documentclass[12pt,letterpaper,twocolumn]{article}

\usepackage{graphicx}
\usepackage{amsmath}
\usepackage{amssymb}
\topmargin=-1in
\evensidemargin=0in
\oddsidemargin=0in
\textwidth=6.5in
\textheight=9.0in
\headsep=0.25in

\begin{document}

\title{Hoja de trabajo No.4}
\author{Sébastien Escobar}
\maketitle

\subsection*{Informatica 1}
\renewcommand{\abstractname}{Ejercicio No.1 }
\begin{abstract}
\textbf{A continuaci\'on se le presentara una serie de definiciones de conjuntos pertenecientes al
conjunto $2^{\mathbb{N}}$. Indicar que definiciones corresponden al mismo conjunto, es decir
que definiciones definen conjuntos que tienen los mismos elementos.}
\\
\begin{enumerate}
        \item{$a:=\{1,2,4,8,16/,32,64\}$} 
        \\Si pertenece al conjunto  $2^{\mathbb{N}}$
        \item{$b:=\{n\ \in \mathbb{N}\ |\ \exists x \in \mathbb{N}\ .\ x=n/5 \}$}
        \\ No pertenece al conjunto  $2^{\mathbb{N}}$
        \item{$c:=\{n\in \mathbb{N}\ |\ \exists x\in\mathbb{N}\ .\ n=x*x \}$} 
        \\No pertenece al conjunto  $2^{\mathbb{N}}$
        \item{$d:=\{n\in\mathbb{N}\ |\ \exists i\in\mathbb{N}\ .\ n=2^i\wedge n<100 \}$} 
        \\No pertenece al conjunto  $2^{\mathbb{N}}$
        \item{$e:=\{ n\in\mathbb{N}\ |\ \exists x\in \mathbb{N}\ .\ x=\sqrt{n} \}$} 
        \\No pertenece al conjunto  $2^{\mathbb{N}}$
        \item{$f:=\{ n\in\mathbb{N}\ |\ \exists x\in \mathbb{N}\ .\ n=x+x+x+x+x \}$} 
        \\No pertenece al conjunto  $2^{\mathbb{N}}$
\end{enumerate}

\end{abstract}

\renewcommand{\abstractname}{Ejercicio No.2 }
\begin{abstract}
\textbf{1. El conjunto de todos los naturales divisibles dentro de 5}
\\
\\Existen dos soluciones para encontrar el conjunto que cumpla con estos terminos:
\\El primero de ellos es saber si un numero es divisible entre otro
\\
\\Para la divisibilidad entre 5, se debe tener que n, tiene que terminar en 5 o 0
\\Ya que de esta manera el residuo sera de 0, y esto demuestra que es divisible
\\
\\Ya que tenemos estas reglas para los números divisibles dentro de 5, expresamos el conjunto como
\\
$A: =\{n \in \mathbb{N}\ |\ \exists x \in \mathbb{N}\ .\ n = x/5\}$
\\
\\Donde i es el residuo de la división anterior. Si i es 0, el numero n es divisible entre su divisor 5, y por lo tanto pertenece al conjunto.
\\
\\\textbf{2. El conjunto de todos los naturales divisibles dentro de 4 y 5}
\\
\\Aplicando la misma definición de divisibilidad todo número que sea divisible entre 4 y 5 simultáneamente pertenece al conjunto.
\\
\\Con un pequeño grado de conocimiento podemos concordar en que el primer numero divisible entre 4 y 5 es la multiplicación de estos, lo que da como resultado 20, por lo tanto, todo múltiplo de 20 cumple con el conjunto antes definido.
\\
\\$A: = \{n \in \mathbb{N}\ |\ x \in N. n = x/4\} 
\\ B: = \{n \in \mathbb{N}\ |\ x \in N. n = x/5\}$
\\
\\Sea i el residuo de la división del conjunto A, y u el residuo de la división del conjunto B
\\
\\Si, i = 0, y, simultáneamente u = 0, entonces el numero x es perteneciente al conjunto de todos los naturales divisibles dentro de 4 y 5
\\
\\\textbf{3. El conjunto de todos los naturales que son primos}
\\
\\Definimos a un numero primo, como aquel numero natural que es divisible únicamente entre 1 y el mismo. Por lo tanto, para conocer si un numero es primo se debe de dividir el numero entre todos sus antecesores. 
\\
\\$A: = \{n \in \mathbb{N}\ |\ \exists x \in \mathbb{N}\ .\ n = x / [(x), (x-1), ((x-1)-1) … 1]\} $
\\
\\Sea i el residuo de la operación anterior, si existe una operación donde el i es 0, dicho numero x no pertenece al conjunto de los números primos. 
\\
\\Por el contrario, si se efectúa la operación anterior y en ningún momento
\\
\\ i = 0, y n = 1 cuando x/x, y n=x 
\\
\\cuando x/1, entonces dicho numero x es perteneciente al conjunto de los primos.
\\
\\\textbf{4. El conjunto de números naturales que contienen un numero divisible dentro de 15}
\\
\\Regresando con nuestra definición de divisible, conoceremos que todo numero con un residuo 0 al ser dividido entre 15, es perteneciente a nuestro conjunto.
\\
\\Por consiguiente, todo número natural multiplicado por 15, es perteneciente ha dicho conjunto, por lo que se puede definir de dos maneras.
\\
\\$A: = \{n \in \mathbb{N}\ |\ \exists x \in \mathbb{N}\ .\ n = x/15 \}$
\\ 
\\Sea i el residuo de la operación anterior, si i = 0, el numero x es perteneciente a nuestro conjunto. 
\\
\\Por otro lado, podemos conocer los números que pertenecen a dicho conjunto si multiplicamos los números naturales enteros (1,2,3,4,5,6…) * 15, lo que nos daría (15, 30, 45, 60…) los cuales son pertenecientes a nuestro conjunto.
\\
\\\textbf{5. El conjunto de números naturales que al ser sumados producen 42 como resultado}
\\
\\Podemos empezar descomponiendo al numero que queremos llegar en todos sus predecesores, siendo estos (1,2,3,4,5,6…. 40,41) 
\\
\\Para nuestro primer conjunto podríamos definir n = 1 y m = 41, la suma de n + m=42. Si a n le sumamos 1 y  m le restamos 1. 
\\
\\Nuestra expresión quedaría intacta, y el resultado seguiría siendo 42.
\\
\\$A: = \{n \in \mathbb{N}\ |\ \exists m \in \mathbb{N}\ .\ [(n + 1) + (m – 1)]\}$
\\ Donde n =1 y m = 41.
\\
\\Nuestro siguiente conjunto será relacionado con los números que pueden conformar al 42, siendo estos (42,21,14,7,6,3,2,1), para obtener estos números se aplicó, 
\\
\\$B: = \{n \in \mathbb{N}\ |\ \exists x \in \mathbb{N}\ .\ n = 42/x\}$
donde i es el residuo y si este es 0, entonces n es parte del conjunto.
\\
\\ La suma de n veces del primer número de dicho conjunto es igual a 42, donde n es el ultimo número si u es la posición en dicho conjunto podemos definir u1 = 1 como 42 y u2 = 8 como 1.
\\
\\$C: = \{n \in \mathbb{N}\ |\ \exists m \in \mathbb{N}\ .\ 42 = [(u1 + 1)  + ((u2 veces u1) - 1)]\}$ 
\end{abstract}

\renewcommand{\abstractname}{Ejercicio No.3 }
\begin{abstract}
Definimos un número semi primo como aquel numero el cual es el producto de dos números primos y que tienen la peculiaridad de ser divisible únicamente por dichos primos que se multiplicaron y uno.
\\
\\$N30 := \{n \in \mathbb{N}\ |\ n \leq 30\}$ es la expresión que relaciona a todos los números semi-primos menores a 30 con los números primos que lo forman.
\\
\\Por ello hay que definir a todos los números primos menores a 30, siendo estos: 
\\
\\{2   3   5   7   11   13   17   19   23   29}
\\
\\Por consiguiente, empezamos a crear nuestros números semi primos
\\ $\langle2, 3, 6\rangle \langle2, 5, 10 \rangle \langle2, 7, 14 \rangle \langle2, 11, 22 \rangle \langle2, 13, 26 \rangle
\\ \langle3, 5, 15 \rangle \langle3, 7, 21 \rangle$
\\
\\Siendo estos tripletes donde los primeros dos números, son primos, lo cuales al multiplicar dan como resultado el 3 número siendo estos los semi primos

\end{abstract}

\renewcommand{\abstractname}{Ejercicio No.4 }
\begin{abstract}
\textbf{Utilicé la jerga matemática para definir los conjuntos a los que corresponden las siguientes funciones:}
\\
\\ \textbf{1. $f:\mathbb{N}\rightarrow\mathbb{N}$; $f(x)=x+x$ }
\\
\\Sea f una función donde entra un numero natural y sale como resultado otro número natural. 
\\
\\Por lo tanto, x es un numero perteneciente al conjunto de N, y f(x) de la misma manera es perteneciente.
\\
\\Nuestro caso base podría ser f (1) = 1 + 1, dando como resultado 2, todos pertenecientes a N
\\
\\ \textbf{2. {$g:\mathbb{N}\rightarrow\mathbb{B}$; $g(x)$ es verdadero si
        $x$ es divisible dentro de $5$, falso en caso contrario. Nota: $\mathbb{B}=
        \{\mathtt{true},\mathtt{false}\}$, puede definir dos conjuntos separados y
        definir la funci\'on como la union de ambos conjuntos.}}
\\
\\Sea A un conjunto de números para los cuales, g(x) como resultado de true, lo cual corresponde a todos los números que sean divisibles dentro de 5.
\\
\\Esto puede expresarse como todos los múltiplos de 5, o como todos aquellos números que terminen en 5 o en 0.
\\
\\Por lo tanto, A: {5, 10, 15, 20, 25, 30, 35, 40 … etc.}
\\Los cuales dan como resultado de g(x) = true
\\
\\Por otro lado, sea B un conjunto de números para los cuales, g(x) como resultado de false, lo cual corresponde a todos los números que no sean divisibles dentro de 5.
\\
\\Para lo cual se podría tomar un conjunto universo de todos los N, y sustraer el conjunto A, lo que nos daría nuestro conjunto B.
\\
\\De otra manera podría ser B, todos aquellos números que no terminen en 5 o 0, lo que daría nuestro conjunto. 
\\
\\Por consiguiente, el conjunto que corresponde a la función anterior es la unión de los conjuntos A y B. 
\\
\\ \textbf{3. {Indicar el conjunto al que pertenece la funci\'on $f\circ g$}}
\\
\\f $\circ$ g pertenece al conjunto N
\\
\\ \textbf{4. {Definir el conjunto que corresponde a la funci\'on $f\circ g$}}
\end{abstract}

\renewcommand{\abstractname}{Ejercicio No.5 }
\begin{abstract}
{$f(x)=x^2$} es surjetiva
\\
\\{$g(x)=\frac{1}{cos(x-1)}$} es injectiva
\\
\\{$h(x)=2x$} es bijectiva
\\
\\{$w(x)=x+1$} es bijectiva
\end{abstract}

\renewcommand{\abstractname}{Ejercicio No.6 }
\begin{abstract}
\textbf{A continuaci\'on se definira una bijecci\'on entre los numeros naturales ($\mathbb{N}$) y los
numeros enteros ($\mathbb{Z}$). Se utilizaran varios conjuntos intermediariarios para facilitar
el proceso.}
\\
\\{Definir el conjunto $B_1\in \mathbb{N}\times\mathbb{N}$ el cula empareja a los
        numeros naturales \emph{pares} con todos los naturales mayores a $0$. Eg. $B_1=\{
        \langle 2,1 \rangle, \langle 4,2 \rangle, \langle 6, 3 \rangle\ldots \}$}
\\
\\{Definir el conjunto $B_{2a}\in \mathbb{N}\times\mathbb{N}$ el cula empareja a los
        numeros naturales \emph{impares} con todos los naturales mayores a $0$. Eg. $B_{2a}=\{
        \langle 1,1 \rangle, \langle 3,2 \rangle, \langle 5, 3 \rangle\ldots \}$}
\\
\\{Definir el conjunto $B_{2}\in \mathbb{N}\times\mathbb{Z}$ el cual se definie
        exactamente igual al conjunto $B_{2a}$ excepto que los valores en el contradominio
        son negativos}
\\
\\{El conjutno $B:= \{\langle 0,0\rangle \}\cup B_{1} \cup B_{2}$ es la bijeccion
        que se intenta definir.}

\end{abstract}
\end{document}

